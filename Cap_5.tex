\chapter{Conclusions}
\label{chap:Conclusions}

This work demonstrates great advancements in the area of Vehicle Re-Identification by extensive research and implementation of various architectural and methodological enhancements. The results show a huge impact on the performance of Re-Identification by architectures based on Instance-Batch Normalization (IBN). IBN was incorporated into the backbone network and, more specifically, the ResNet50-IBN-a architecture, which showed a remarkable improvement over the baseline ResNet-50 model on the VeRi-776 dataset with a large increase in mAP from 70.40\% to 77.19\%, with the best performing model, through various optimizations and tricks, reaching a total score of 81.33\% points.
The complex combination of architecture components was the key to our success, facilitated by the implementation of BNNeck and a Warmup Cosine scheduling strategy, which demonstrated consistent and superior performance improvements across various configurations. Moreover, our loss function approach, combining Triplet loss and Cross Entropy loss, contributed meaningfully to the discrimination of feature space between vehicle clusters, especially after careful tuning of the Triplet loss margin to 0.30, which yielded the best results.

In the context of Multi-Target Multi-Camera (MTMC) tracking, our research revealed several critical insights. The tracking algorithm itself proved to be an integral and fundamental part of the MTMC pipeline and played a very important role in the overall performance of the system. Our Custom Tracker combined with YOLOv11x achieved a very strong result, with an IDF1 score of 85.0\% and MOTA of 95.6\%, outperforming BOTSort and ByteTrack alternatives. This finding highlights the vital importance of choosing suitable trackers to maintain stable object identity across different cameras. The experimental results also showed the need for optimizing the system in a balanced way. While lightweight versions of YOLO improved computational efficiency, they suffered significant performance degradation, with IDF1 scores dropping as low as 68.5\%. This tradeoff between computational resources and tracking accuracy is an essential consideration for practical applications.

Our findings open several promising directions for future research in vehicle re-identification and MTMC systems. The architecture optimization pathway suggests exploring the scalability of IBN-enhanced networks across various model depths with particular attention to the balance between computational complexity and performance gains. This investigation should extend beyond ResNet variants to examine alternative backbone networks while keeping the proven effectiveness of BNNeck and Warmup Cosine scheduling.
The development of loss functions is a topic that needs to be explored further, in fact, the next research works should focus on advanced combinations of loss functions and weighting methods, especially adaptive margin methods related to Triplet loss. This line of investigation might overcome the present limitations in feature discrimination while further increasing the robustness of vehicle matching under changing environmental conditions.
A more extensive exploration of the interactions between Momentum Adaptive Loss Weight (MALW), different architectural frameworks, and various loss functions represents a promising research direction. In the final analysis, the integration of a more robust base network, linked with advanced loss functions, feature-based strategies (merging local and global features with attentions modules), and adaptive weighting strategies could significantly enhance the performance of Vehicle Re-Identification systems, particularly in challenging real-world scenarios.

The tracking component inherent in MTMC systems certainly merits deeper investigation, given its proven importance in our work. Future research should aim at developing more robust tracking algorithms that can maintain consistent vehicle identification throughout complex camera networks while decreasing computational requirements. This includes looking into hybrid tracking methods that will combine the advantages of different paradigms of tracking, as well as exploring a potential adaptive tracking strategy that can adjust to varying environmental conditions. Also, explicitly integrating Re-ID features into the tracking pipeline—a step not considered at the present stage—could potentially enhance tracking performance by improving the accuracy of vehicle matching across cameras. One could also directly use the Re-ID features initially computed to by the Re-ID model, without the need for a re-computation of the features in the tracking pipeline.
Finally, the integration of more sophisticated filtering mechanisms into the MTMC pipeline is one of the important directions for future research. While our current filtering technique showed some promising results in reducing computational requirements, there is still a wide margin for potentially developing more intelligent filtering schemes that can maintain high levels of accuracy while, at the same time, optimizing system resources.
