\begin{abstract}
    
    Multi-Camera Vehicle Re-Identification (Re-ID) is a core task of Intelligent Transportation Systems (ITS) and urban surveillance, which aims at connecting the same vehicle across different non-overlapping camera views. Although significant progress has been achieved with the introduction of Deep Learning-based Feature Extraction models and Object-Detection models, vehicle Re-ID has its own peculiar challenges. These include occlusion, viewpoint changes, illumination variation and potential ID-switch errors introduced by the object trackers. However, reaching a robust vehicle matching in a large camera network requires balancing computational efficiency and scalability, along with coherent and consistent trajectories.

    In this study, a novel framework for efficient Multi-Target Multi-Camera Tracking (MTMC) is proposed. Specifically, this work focuses on combining an optimal object detection model, namely, YOLO, for vehicle detection and tracking, with an advanced and complex feature extraction protocol using ResNet-based architectures with different optimizations (IBN family) and methods for generating accurate embeddings of vehicle appearances. For each vehicle observed through a single camera view, the pipeline performs trajectory generation which are successively connected across cameras using similarity-based matching methods. Several filtering steps are also introduced to enhance the reliability of bounding boxes by removing stationary, incomplete or color-mismatched vehicles, helping in the reduction of possibly wrong tracklets connections.

    Given the complexity of large-scale multi-camera networks, a key focus of this work is on refining vehicle matching by using mean embedding comparisons and spatio-temporal clustering constraints to unify the different tracklets into a common one. The system is evaluated on the AICity 2022 Track 2 Dataset, a benchmark for MTMC, with competitive results in terms of IDF1, MOTA, and overall Re-Identification accuracy. In addition, an optimized and scalable MongoDB storage algorithm or a simple Pickle system, depending on the user, makes its own way to ensure that database operations work effectively, ensuring, as well, an appropriate retrieval of vehicle trajectories and bounding boxes.

    This paper aims to give a solution that is scalable, accurate, and computationally feasible for real-world applications in smart cities and traffic analysis by addressing both appearance-related and temporal elements of Vehicle Re-Identification.
    
\end{abstract}
